%%%%%%%%%%%%%%%%%%%%%%%%%%%%%%%%%%%%%%%%%
% Twenty Seconds Resume/CV
% LaTeX Template
% Version 1.0 (14/7/16)
%
% Original author:
% Carmine Spagnuolo (cspagnuolo@unisa.it) with major modifications by 
% Vel (vel@LaTeXTemplates.com) and Harsh (harsh.gadgil@gmail.com)
%
% License:
% The MIT License (see included LICENSE file)
%
%%%%%%%%%%%%%%%%%%%%%%%%%%%%%%%%%%%%%%%%%

\documentclass[letterpaper]{twentysecondcv} % a4paper for A4

\usepackage[spanish]{babel}

% Burbujas de habilidades
\newcommand\skills{ 
~
    \smartdiagram[bubble diagram]{
        \textbf{Desarrollo}\\\textbf{Web},
        \textbf{Ciencia de}\\\textbf{Datos},
        \textbf{Optimización}\\ \textbf{Numérica},
        \textbf{Desarrollo}\\\textbf{Móvil},
        \textbf{Machine}\\\textbf{Learning},
        \textbf{~~~Sistemas~~~}\\\textbf{~~Embebidos~~},
        \textbf{Diseño}\\\textbf{3D}
    }
}

% Barras de conocimiento de lenguajes
\programming{{OpenSCAD \textbullet {\large\LaTeX} \textbullet Python / 3.5}, {Pascal \textbullet Visual Basic \textbullet Java / 3}, {C++ \textbullet JavaScript \textbullet Matlab / 5}}

% Cursos realizados
\courses{
    \textbf{Minería de Datos y Aprendizaje Automatizado.}\\ Duración: 90hs.  Nota obtenida: 10 (diez).\\
    \textbf{Tópicos en Big Data.}\\ Duración: 80hs.  Nota obtenida: 10 (diez).\\
    \textbf{Modelos Matemáticos de Simulación en la Investigación Agropecuaria.}\\ Duración: 60hs.  Nota obtenida: 10 (diez).\\
    \textbf{Sistemas Distribuidos de Tiempo Real.}\\ Duración: 60hs.  Nota obtenida: 10 (diez).\\
    \textbf{Internet de las Cosas y el Control, un Enfoque Holístico.}\\ Duración: 60hs.  Nota obtenida: 10 (diez).\\
}


%----------------------------------------------------------------------------------------
%    INFORMACION PERSONAL
%----------------------------------------------------------------------------------------

\cvname{MATÍAS MICHELETTO}
\cvjobtitle{ Ingeniero Electrónico \\ Desarrollador de software }
\cvlinkedin{in/matiasmicheletto}
\cvgithub{matiasmicheletto}
\cvnumberphone{+54 9 291 459 5181}
\cvsite{www.diec.uns.edu.ar/rts}
\cvmail{matias.micheletto@uns.edu.ar}
\cvfacebook{/miche1989}
\cvinstagram{/matias.jm_}
\cvyoutube{/channel/UCI0mjA0Hl7jok8DzrWavTng}

%----------------------------------------------------------------------------------------

\begin{document}

\makeprofilefirst % Sidebar de primera pagina

\section{Educación}
\begin{twenty}
    %\twentyitem{<dates>}{<title>}{<organization>}{<location>}{<description>}
    \twentyitem
        {2006 - 2016}
        {}
        {Ingeniería Electrónica.}
        {Promedio: 8.17/10.0}
        {\href{http://www.uns.edu.ar/}{Universidad Nacional del Sur.}}
        {}
    \twentyitem
        {2004 - 2006}
        {}
        {Cs. Naturales y Téc. en Prod. Agropecuaria.}
        {Promedio: 7.79/10.0}
        {Escuela de Agricultura y Ganadería U.N.S.}
        {}
\end{twenty}

\section{Docencia}
\begin{twenty}
    \twentyitem
        {2016 - pres.}
        {}
        {Asistente de cátedra dedicación simple}
        {\href{http://www.diec.uns.edu.ar/}{DIEC - UNS}}
        {}
        {Diseño de Circuitos Lógicos - Técnicas de Diseño Digital}
\end{twenty}

\section{Publicaciones}
{\begin{itemize}
\item {\bf Utilizando UML para el aprendizaje del modelado y diseño de sistemas ciber-físicos.} \textit{Leonardo Ordínez, Rodrigo Santos, Gabriel Eggly, Matías Micheletto.}
IEEE-RITA, 14(3), -.
\item {\bf Scheduling Mandatory-Optional-Time Tasks in Homogeneous Multi-Core Systems with Energy Constraints Using Bio-Inspired Meta-Heuristics.} \textit{Matías Micheletto, Rodrigo Santos, Javier Orozco.} Journal of Universal Computer Science, 25(4), 390-417.
\item {\bf Flying Real-Time Network to Coordinate Disaster Relief Activities in Urban Areas.} \textit{Matías Micheletto, Vinicius Petrucci, Rodrigo Santos, Javier Orozco, Daniel Mosse, Sergio Ochoa, Roc Meseguer.} Sensors, 18(5), 1662.
\item {\bf Real-Time Communication Support for Underwater Acoustic Sensor Networks.} \textit{Rodrigo Santos, Javier Orozco, Matías Micheletto, Sergio Ochoa, Roc Meseguer, Pere Milan, Carlos Molina.} Sensors, 17(7), 1629.
\end{itemize}}

\vspace{2mm}

\section{Becas}
\begin{twenty}
    \twentyitem
        {2016 - pres}
        {}
        {Beca de Posgrado}
        {ICIC-CONICET}
        {\textit{``Desarrollo de Sistemas Computacionales para Sustento de Tareas Agropecuarias en el Sudoeste de la Provincia de Buenos Aires''.}}
        {\textbf{Director:} Dr. Rodrigo Santos. \textbf{Co-Director:} Dr. Juan Galantini.} \\
    \twentyitem
        {2015 - 2016}
        {}
        {Beca de Investigación}
        {Secretaría General de Ciencia y Tecnología U.N.S.}
        {PGI-MAyDS \textit{``Sustentabilidad de Construcciones Civiles''.}}
        {\textbf{Director:} Dr. Néstor Ortega.} \\
    \twentyitem
        {2014}
        {}
        {Beca de Introducción a la Investigación para Alumnos Avanzados}
        {}
        {\textit{``Planificación Óptima de Sistemas de Tiempo Real en Plataformas Multicore''.}}
        {\textbf{Director:} Dr. Javier Orozco.} \\
    \twentyitem
        {2014}
        {}
        {Beca Interna de Estímulo al Estudio}
        {DIEC - UNS}
        {}
        {} \\
    \twentyitem
        {2013}
        {}
        {Beca Interna de Estímulo al Estudio}
        {DIEC - UNS}
        {}
        {}
\end{twenty}

\section{Premios}
\begin{twenty}
    \twentyitem
        {2018}
        {}
        {Agroton 2018}
        {Segundo premio}
        {\textbf{Equipo A:} Matías Micheletto, Alejandro André, Matías Timi, José Augusto Strick, Franco Tronelli y Guido Temperini.}
        {\textbf{Propuesta:} Paquete de Automatización e Información Integral de Soluciones Agropecuarias (PAIISA)}
\end{twenty}



%----------------------------------------------------------------------------------------
%    PAGINA 2
%----------------------------------------------------------------------------------------

\newpage

% Barras de conocimiento de tecnologías
\techs{{Firebase \textbullet NodeJS \textbullet Apache Cordova / 4}, {Angular \textbullet Framework7 / 3.5}, {Materialize \textbullet Bootstrap / 4}, {Matlab \textbullet Simulink \textbullet Octave / 4.5}, {Arduino \textbullet Processing \textbullet Eclipse / 5}}

% Proyectos realizados
\projects{
    \textbf{\href{http://webcapp.com}{CAPP Backoffice:}} Panel de administración de E-Commerce para corralones y negocios de venta de materiales de construcción.\\
    \textbf{\href{http://www.diec.uns.edu.ar/rts}{RTS Research Group:}} Sitio web dinámico para grupo de investigación.\\
    \textbf{\href{https://sebastianlopezrepresentaciones.com}{Sebastian López Representaciones:}} \\PWA con panel de administrador para CRM y red de clientes, para compañía distribuidora de materiales de construcción. \\
    \textbf{\href{http://www.cipressus.uns.edu.ar}{Cipressus:}} Sistema de gestión de contenidos para el aprendizaje de sistemas digitales. \\
    \textbf{\href{http://www.psimesh.com}{PsiMESH:}} Plataforma con actividades digitales de evaluación neuro psicológicas. \\
    \textbf{\href{https://play.google.com/store/apps/details?id=com.inta.criollo2&hl=es}{Criollo:}} Aplicación nativa Android para cálculos de pulverizaciones agrícolas. 
}

\makeprofilesecond % Sidebar de segunda pagina

\section{Experiencia}

\begin{twenty} 
\twentyitem
        {2017 - pres.}
        {}
        {Web developer - IoT developer}
        {\href{http://www.cinaweb.org/}{CINA (Neufitech S.R.L.)}}
        {}
        {\begin{itemize}
        \item Desarrollo de plataforma web para aplicaciones de actividades neuro-psicológicas (PsiMESH).
        \item Sistema de solicitud y administración de turnos Centro Integral de Neurociencias Aplicadas.
        \item Diseño e implementación de dispositivos IoT para asistencia en comunicación y movilidad de personas con discapacidades motrices.
        \end{itemize}}
        \\
\twentyitem
        {2019 - pres.}
        {}
        {Web developer}
        {\href{http://www.webcapp.com/}{CAPP (CAPP S.R.L.)}}
        {}
        {\begin{itemize}
        \item Participación en el desarrollo de Back Office y Marketplace para clientes de CAPP.
        \end{itemize}}
\end{twenty}

\section{Participación en congresos}

\begin{twenty}
\twentyitem
    {2019}
    {}
    {48º Jornadas Argentinas de Informática\\} 
    {\textit{``Internet de las Cosas como Bien Social''.}}
    {Gabriel Eggly, Mariano Finochietto, Matías J. Micheletto, Rodrigo Santos.}
    
\twentyitem
    {2018}
    {}
    {47º Jornadas Argentinas de Informática\\}
    {\textit{``Diseño e Implementación de un Escáner Lidar para Análisis Tridimensional de Covertura Vegetal de Suelos''.}}
    {Matías J. Micheletto, Rodrigo Santos, Luciano Zubiaga, Juan Galantini.}

\twentyitem
    {2017}
    {}
    {XXV Jornadas de Jóvenes Investigadores\\}
    {\textit{``Planificación Óptima de un Sistema Multiprocesador de Tiempo Real con Restricciones de Precedencia, Comunicación y Energía''.}}
    {Matías J. Micheletto, Alejandro Borghero, Gabriel M. Eggly.}
    
\twentyitem
    {2017}
    {}
    {46º Jornadas Argentinas de Informática\\}
    {\textit{``Desarrollo de una Aplicación Móvil para Cálculos de Pulverizaciones Agrícolas''.}}
    {Gabriel M. Eggly, Matías J. Micheletto, Juan P. D'Amico, Santiago J. Crocioni.}
    
\twentyitem
    {2016}
    {}
    {V Congreso Internacional sobre Cambio Climático y Desarrollo Sustentable\\}
    {\textit{``Afectación de la Luna en la Medición de la Contaminación Lumínica''.}}
    {Luciana C. Lambertucci, Matías J. Micheletto, Jorge A. Starobinsky, Néstor F. Ortega.}
    
\twentyitem
    {2016}
    {}
    {10$^{th}$ International Conference on Ubiquitous Computing \& Ambient Intelligence\\}
    {\textit{``Scheduling Real-Time Traffic in Underwater Acoustic Wireless Sensor Networks''.}}
    {Rodrigo Santos, Javier Orozco, Matías Micheletto, Sergio Ochoa, Roc Meseguer, Pere Millan, Carlos Molina.}
    
\twentyitem
    {2016}
    {}
    {45º Jornadas Argentinas de Informática\\}
    {\textit{``Diseño e Implementación de un Registrador de Esfuerzos para Maquinaria Agrícola''.}}
    {Matías J. Micheletto, Gabriel M. Eggly, Rodrigo Santos.}
    
\twentyitem
    {2015}
    {}
    {V Brazilian Symposium on Computing Systems Engineering\\}
    {\textit{``Using bioinspired meta-heuristics to solve reward-based energy-aware mandatory/optional real-time tasks scheduling''.}}
    {Matías J. Micheletto, Javier D. Orozco, Rodrigo Santos.}
    
\twentyitem
    {2014}
    {}
    {Simposio Argentino de Sistemas Embebidos\\}
    {\textit{``Design and Implementation of an Embedded Prototype for Monitoring a Combine Harvester''.}}
    {Ana S. Arauz Lozano, Matías J. Micheletto, Leonardo Ordinez, Rodrigo Santos.}
    
\end{twenty}

\end{document} 
